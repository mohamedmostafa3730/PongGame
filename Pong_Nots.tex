(1) what I learn from Pong Program? 
1- Movement;
2- Controls;
3- Collision Detdction;
4- Scoring;
5- Artificial Intelligence;
=========================================================
(2) Steps to develop Pong: 
1- Create a blank screen & game loop;
2- Draw the paddles and the ball;
3- Move the ball around;
4- check for a Collision with all paddles;
5- Move the player's paddle;
6- Move the CPU paddle with Artificial Intelligence;
7- check for a Collision with paddles;
8- Add Scoring;
=========================================================
(3-1) [Step #1] Create a blank screen & game loop: 
every game Structure -> Definitions | Game loop ;
this game -> Definitions -> defining the variables needed | Creatting the game objects ;
          -> Game loop -> Updating the positions of the game objects | checking for collisions ;
=========================================================
(3-2) [Step #2] Draw the paddles and the ball: 
How to draw with [Raylib] => { Rectangle - Circle - Line - Poly }
 1- void DrowRectangle(int posX, int posY, int height, Color color);
 2- void DrowCircle(int centerX, int centerY, float radius, Color color);
 3- void DrowLine(int startPosX, int startPosY, int endPosx, int endPosY, Color color);
 4- void DrowPoly(Vector2 center, int sides, float radius, float rotation, Color color); 

=========================================================
(3-3) [Step #3] Move the ball around: 
  In this Step I make a Ball Class to then, used functions sach as : 
  1- void DrowCircle(int centerX, int centerY, float radius, Color color); // in raylib Library //
  2- ClearBackground(Color color); // in raylib Library //
  3- Draw(); 
  4- Update();
=========================================================
(3-4) [Step #4] check for a Collision with all paddles:
 Add if condation in Update() and, used functions sach as:
 1- GetScreenHeight();   // in raylib Library //
 2- GetScreenWidth();   // in raylib Library //


=========================================================
(3-5) [Step #5] Move the player's paddle:
 Make the player class to control of Rectangle & fix some issues,
 use some functions sach as:
 1- IsKeyDown(KEY_UP) // in raylib Library //
 2- IsKeyDown(KEY_DOWN) // in raylib Library //
 3- GetScreenHeight() // in raylib Library //
 3- Draw(); 
 4- Update();
 =========================================================
 (3-6) [Step #6] Move the CPU paddle with Artificial Intelligence:
  Make the CPU paddle class to control of paddle with CPU's AI,
  and, the CPU paddle class is inhertns of paddle class to use some functions,
  and, use some functions sach as:
  1- CPU.Update();
  2- CPU.Draw();
  3- LimitMovement();
  and, use some data sach as:
  1- CPU.width;
  2- CPU.height;
  3- CPU.x;
  4- CPU.y;
  5- CPU.speed;
=========================================================
(3-7) [Step #7] check for a Collision with paddles:
 USE Collision Detection functions:
  1- CheckCollisionCircleRec(Vector2 center, float radius, Rectangle rec);
=========================================================
(3-8) [Step #8] Add Scoring;
 Make 2 variables ( player_score | CPU_score )
 DrawText(text, xpos, ypos, fontsize, color);
 DrawRectangleRounded(rec, roundness, segment, color);
=========================================================